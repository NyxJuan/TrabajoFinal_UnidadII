\section{Marco Teórico}
\vspace{12pt}

\subsection{Definición}
Oracle Database es un sistema de gestión de base de datos de tipo objeto-relacional (ORDBMS, por el acrónimo en inglés de Object-Relational Data Base Management System), desarrollado por Oracle Corporation.\\
Su dominio en el mercado de servidores empresariales había sido casi total hasta que recientemente tiene la competencia del Microsoft SQL Server y de la oferta de otros RDBMS con licencia libre como PostgreSQL, MySQL o Firebird.\\
Las últimas versiones de Oracle han sido certificadas para poder trabajar bajo GNU/Linux.\\


\subsection{Historia}

Oracle surge en 1977 bajo el nombre de SDL (Software Development Laboratories).
En 1979, SDL cambia su nombre por Relational Software, Inc. (RSI).
La fundación de SDL fue motivada principalmente a partir de un estudio sobre los SGBD (Sistemas Gestores de Base de Datos) de George Koch. Computer World definió este estudio como uno de los más completos jamás escritos sobre bases de datos. Este artículo incluía una comparativa de productos que dirigía a Relational Software como el más completo desde el punto de vista técnico. 
La tecnología Oracle se encuentra prácticamente en todas las industrias alrededor del mundo y en las oficinas de 98 de las 100 empresas Fortune 100. Oracle es la primera compañía de software que desarrolla e implementa software para empresas cien por ciento activado por Internet a través de toda su línea de productos: base de datos, aplicaciones comerciales y herramientas de desarrollo de aplicaciones y soporte de decisiones. Oracle es el proveedor mundial líder de software para administración de información, y la segunda empresa de software.
\vspace{12pt}\\
\textbf {Oracle, a partir de la versión 10g Release 2, cuenta con 7 ediciones:}\\
Enterprise Edition (EE).\\
Standard Edition (SE).\\
Standard Edition One (SE1).\\
Standard Edition 2 (SE2).\\
Express Edition (XE).\\
Personal Edition (PE).\\
Lite Edition (LE).\\
\vspace{12pt}\\
